\documentclass[a4paper,12pt]{article}
\usepackage{ucs}
\usepackage[utf8x]{inputenc} % Включаем поддержку UTF8
\usepackage[russian]{babel}  % Включаем пакет для поддержки русского языка
\title{Install Help Guide}
\date{16.04.2013}
\author{Arcocode Team}

%Настройка полей
\usepackage{geometry}
\geometry{left=2cm}% левое поле
\geometry{right=1.5cm}% правое поле
\geometry{top=1cm}% верхнее поле
\geometry{bottom=2cm}% нижнее поле

%\setcounter{tocdepth}{2} 


\begin{document}
\maketitle
\newpage
\textbf{Инструкция по развертыванию сервера.}
1) С http://glassfish.java.net/ скачивается установщик GlassFish Server Open Source Edition [справка, что такое Glassfish — первые два абзаца из Википедии или что-то в этом ключе].

2) Администрирование из командной строки — glassfish3/bin/asadmin.

3) По умолчанию создается домен под названием domain1 [справка, что такое домен]. Командами «asadmin delete-domain domain1» и «asadmin create-domain arcocode-domain» создается новый, с названием arcocode-domain; следует убедиться, что порт администрирования — 4848, http — 8080, https — 8181 [сейчас название и порты жестко прописаны в pom.xml, но можно сделать конфигурируемым, если кому-то надо].

4) Файл из репозитория etc/settings.xml-example следует скопировать как C:\\Users\\.....\\.m2\\settings.xml и поменять там настройки на путь поддиректории glassfish из директории, куда установлен сервер, и соответствующие логин и пароль [не уверен насчет того, куда класть конфигурацию в Windows — maven выводит много отладочной информации по ключу -X, можно проверить].

5) Собранный .war-файл maven устанавливает на сервер по команде «mvn glassfish:deploy» или «mvn glassfish:redeploy», в зависимости от того, первый ли это запуск, или программа уже работает. Остановить программу — «mvn glassfish:undeploy».

6) Проверить работу сервера можно http-запросами к http://localhost:8080/arcocode/api/\{что-то\} — например, плагином Poster к Firefox. В соответствии с API.

\end{document}