\documentclass[a4paper,12pt]{article}
\usepackage{ucs}
\usepackage[utf8x]{inputenc} % Включаем поддержку UTF8
\usepackage[russian]{babel}  % Включаем пакет для поддержки русского языка
\title{Техническое задание группы 3\newline Code City }
\date{20.02.2013}
\author{Xeenon}

%Настройка полей
\usepackage{geometry}
\geometry{left=2cm}% левое поле
\geometry{right=1.5cm}% правое поле
\geometry{top=1cm}% верхнее поле
\geometry{bottom=2cm}% нижнее поле

%\setcounter{tocdepth}{2} 


\begin{document}
\maketitle
\newpage
\textbf{1. Введение}
\newline

	\textbf{1.1. Назначение и область действия}
	
		Данный программный продукт предназначен для отображения метрик кода.
		Согласно заданию, допускается оценка метрик кода хотя бы для 1 языка.
	\newline	

	\textbf{1.2. Краткий обзор}
	
		Программный продукт позволяет загружать информацию из репозитория и совершать оценку кода на основе метрик и представлять ее в виде трехмерного города, 
		чьи размеры будут соответствовать метрикам кода.
\maketitle
\newpage
\textbf{2. Общее описание}
\newline

	\textbf{2.1. Перспектива изделия}
	
		Получение зачета.
		\newline
		
	\textbf{2.2. Функции изделия}
	
		Подключение к репозиторию.
		
		Загрузка кода из выбранного репозитория (при необходимости производится авторизация).
		
		Оценка метрик кода.
		
		Отображение метрик кода в виде города.	
		\newline
		
	\textbf{2.3. Характеристики пользователя}
	
		Требования к навыкам пользователя:
		
		Владение ПК на базовом уровне.
		
		Специальных требований к навыкам пользователя не предъявляется.
		\newline
		
	\textbf{2.4. Ограничения}
	
		Программный продукт оценивает только 1 язык программирования (Java).
		\newline

	\textbf{2.5. Допущения и зависимости}
	
		Клиент-серверная архитектура.
		
		Клиент – java-script(используемый фреймворк), отображение метрик.
		
		Сервер – java(библиотеки), закачка файлов, пересчет метрик.
		\newline
		
	\textbf{2.6. Разделение требований}
	
		\textit{Этап 1}
		
		Козин Георгий - Серверная часть
		
		Минеев Александр - Техническое задание и документация
		
		Тимофеева Юлия - Клиентская часть (HTML) и сборка Maven
		
		Яковлев Сергей - Клиентская часть (WebGL)
		
		Яров Максим - Серверная часть
\maketitle
\newpage
\textbf{3. Требования к внешнему интерфейсу}
\newline

	\textbf{3.1. Пользовательские интерфейсы}
	
	Веб-приложение, которое подключается к серверу.
	
	Пользователь может задавать адрес репозитория, к которому хочет подключиться. При необходимости может ввести логин и пароль для авторизации на данном репозитории. Может просматривать и добавлять в очередь загрузки выбранный проект. Может запускать анализ метрик кода. Может запускать отображение метрик кода.
	\newline
	
	\textbf{3.2. Программные интерфейсы}
	
		\textit{Сервер.}
		
		Получает команды от клиентского приложения. Скачивает выбранный репозиторий. Производит анализ кода на основе метрик. Передает данные клиентскому приложению
		
		REST – используемый REST API
		
		библиотеки для клиентской части(Git, Mercurial, SVN)
	
		Парсер java (package, class,…) или java compiler api или java developer tools.
		
\maketitle
\newpage
\textbf{4. Особенности системы}
\newline

	\textbf{4.1.	Используемые метрики}
	\newline
	
	стандартные метрики (прописать в ТЗ!) 
	
	размер класса (количество строк кода или методов(желательно))
	
	количество атрибутов
	\newline
	
	\textbf{4.2.	Особенности реализации}
	\newline
	
		Разметка по пакетам java.
		
		Дороги между областями пакетов, которые будут отражать степень связей между классами.
		
		Активная камера - поворот, масштабирование, изменение точки обзора.
		
		Интерактивное взаимодействие с объектами: выделение объекта, данные по метрикам.

		Хранение сессии.

		Отображение списка отправленных репозиториев, \% работ; интерфейса для отправки на анализ, типа репозитория, логина, пароля.
		\newline
		
	\textbf{4.3.	Сборка проекта}
	\newline
	
	\textit{Сервер:}
	
	MAVEN
	\newline
	
	\textit{Анализ сборки клиента:}
	
	плагин MAVEN
	\newline
	
	\textit{Сборка документации:}
	
	MAVEN

\maketitle
\newpage
\textbf{5. Нефункциональные требования}
\newline

	\textbf{5.1.	Требования производительности}
	
		Загрузка файлов не должна приводить к сбоям в работе клиентского приложения (превышение максимального времени ожидания).
		\newline
		
	\textbf{5.2.	Требования надежности}
	
		Особые требования не предъявляются.
	\newline
	
	\textbf{5.3.	Требования безопасности}
	
		URL репозитория и пользовательский пароль не должны передаваться в открытом виде.
	\newline

\end{document}